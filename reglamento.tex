\documentclass[12pt, letterpaper]{article}
\usepackage{graphicx}
\usepackage[spanish]{babel}
\usepackage{parskip}
\usepackage{enumitem}


% Document Settings
\setlength{\parindent}{0cm}
\setlength{\parskip}{4mm}
\setlist{nosep}
\renewcommand{\familydefault}{\sfdefault}

% Metadata
\title{Reglamento Interno - HackLab BrickHeads}
\date{Febrero 2023}

% Document
\begin{document}
    \maketitle
    \hspace{1cm}
    \begin{center}
        HackLab BrickHeads es un espacio Hacker de conocimiento abierto, 
        experimentación y aprendizaje colectivo
    \end{center}

    Nuestros principales pilares son:
    \begin{itemize}
        \item \textbf{T}ecnología
        \item \textbf{E}ducación
        \item \textbf{C}ultura/\textbf{C}iencia
        \item \textbf{S}ociedad
    \end{itemize}

    \section{Manifiesto}
    El conocimiento es un pilar fundamental del desarrollo humano, es libre y no
    debería privarse.
    La tecnología es un medio, mas no el fin. Es una herramienta de las muchas 
    para la transformación y evolución a una sociedad mejor.

    La privacidad, la seguridad y los datos personales son derechos humanos
    fundamentales.

    La colaboración, contribución y participación abierta son clave para el
    desarrollo de tecnología con conciencia y responsabilidad social.

    Es importante fomentar la inclusión y diversidad en todos los aspectos.

    Nada es absolutamente cierto, ni absolutamente falso, las respuestas más
    interesantes se encuentran en el terreno de lo matizado.

    Debemos manejar, enseñar y difundir tecnología de forma crítica y reflexiva.
\end{document}